%-----------------------------------------------------------------------------%
\chapter{\topikSatu}
%-----------------------------------------------------------------------------%

%-----------------------------------------------------------------------------%
\section{Eksperimen}
%-----------------------------------------------------------------------------%
Pada bagian ini kami melakukan 2 eksperimen untuk mengamati perilaku \textit{block} dan \textit{thread} pada kernel dan mengamati pengaruh ukuran data terhadap eksekusi \textit{kernel}

\subsection{Eksperimen Perilaku Thread dan Block} 

\todo{}

\subsubsection{Deskripsi Program}

\todo{}

\subsubsection{Hasil Eksperimen}

\todo{}

%\begin{figure}
%	\centering
%	\includegraphics[width=1\textwidth]
%	{pics/process_topologies_demo}
%	\caption{Contoh eksekusi program demo proses topologi}
%	\label{fig:process_topologies_demo}
%\end{figure}  
%
%\begin{figure}
%	\centering
%	\includegraphics[width=1\textwidth]
%	{pics/process_topologies_creation_nbcr}
%	\caption{Eksperimen waktu pembuatan topologi kartesian di cluster UCSD}
%	\label{fig:process_topologies_creation_nbcr}
%\end{figure}  

\subsection{Eksperimen Perilaku Thread dan Block} 

\todo{}

\subsubsection{Deskripsi Program}

\todo{}

\subsubsection{Hasil Eksperimen}

\todo{}

%-----------------------------------------------------------------------------%
\section{Kesimpulan}
%-----------------------------------------------------------------------------%

\todo{}