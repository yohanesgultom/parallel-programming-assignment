%-----------------------------------------------------------------------------%
\chapter{\lingkungan}
%-----------------------------------------------------------------------------%

\section{Lingkungan Pengembangan}

Program paralel yang digunakan di dalam eksperimen ini dibuat menggunakan bahasa C yang di-\f{compile} menggunakan pustaka OpenMPI pada sistem operasi Linux Ubuntu dan Mint.

Kode program-program dan laporan eksperimen ini kami simpan menggunakan layanan GitHub di alamat \f{https://github.com/yohanesgultom/parallel-programming-assignment}. Hal ini kami lakukan untuk mempermudah kolaborasi dalam pembuatan program dan laporan.

\section{Lingkungan Percobaan}

Dalam eksperimen yang dilakukan, kelompok kami menggunakan mesin-mesin dengan GPU NVIDIA dan \cluster CPU Rocks University of California San Diego (UCSD).

%-----------------------------------------------------------------------------%
\subsection{Server GPU Fasilkom Universitas Indonesia (UI)}
%-----------------------------------------------------------------------------%
\textit{Server} dengan GPU NVIDIA merupakan lingkungan utama percobaan ini. Kelompok kami melakukan percobaan di 2 buah server Fasilkom UI:

\begin{enumerate}
	\item Server GTX 980 (152.118.31.27)
	\begin{itemize}
		\item NVIDIA GTX 980 2048 CUDA Cores 4 GB GRAM
		\item Intel(R) Core(TM) i7-3770 CPU 4 cores @ 3.40GHz 
		\item RAM 2x8 GB DDR3 1600 Mhz
		\item SSD SAMSUNG MZ7TD128 128 GB		
		\item OS Debian 7 Wheezy x64
		\item CUDA 7.0
		\item Amber 14 dan Amber Tools 15
	\end{itemize}
	\item Server GTX 970 (152.118.31.34)
	\begin{itemize}
		\item NVIDIA GTX 970 1664 CUDA Cores 4 GB GRAM
		\item Intel(R) Core(TM) i7-3770 CPU 4 cores @ 3.40GHz
		\item RAM 2x8 GB DDR3 1600 Mhz
		\item SSD SAMSUNG MZ7TD128 128 GB		
		\item OS Debian 7 Wheezy x64
		\item CUDA 7.0
		\item Amber 14 dan Amber Tools 15
	\end{itemize}
\end{enumerate}

Semua \textit{server} ini dapat diakses dari jaringan Fasilkom menggunakan protokol \textit{SSH} dan \textit{credential Single Sign On} (SSO) UI. Sedangkan dari luar jaringan UI, semua \textit{server} ini dapat diakses dengan masuk lebih dahulu ke \url{kawung.cs.ui.ac.id} menggunakan protokol SSH juga.

%-----------------------------------------------------------------------------%
\subsection{Personal Computer (Laptop)}
%-----------------------------------------------------------------------------%

Sebagai bahan perbandingan, kami juga menggunakan PC (\textit{notebook}) yang juga menggunakan GPU NVIDIA yang mendukung CUDA:
\begin{itemize}
	\item NVIDIA 940M 384 CUDA Cores 2 GB GRAM
	\item Intel(R) Core(TM) i7-5500U CPU 2 cores @ 3.40GHz
	\item RAM 8 GB DDR3 1600 Mhz
	\item SSD Crucial 250 GB		
	\item OS Ubuntu 15.10 x64
	\item CUDA 7.5
	\item Amber 14 dan Amber Tools 15
\end{itemize}

%-----------------------------------------------------------------------------%
\subsection{Cluster Rocks University of California San Diego (UCSD)}
%-----------------------------------------------------------------------------%
\Cluster Rocks\footnote{www.rocksclusters.org} milik University of California San Diego (UCSD) ini dapat diakses pada alamat \f{nbcr-233.ucsd.edu} menggunakan protokol SSH dari komputer yang telah didaftarkan \f{public key} nya. Berdasarkan informasi dari aplikasi \f{monitoring} Ganglia \footnote{http://nbcr-233.ucsd.edu/ganglia}, cluster ini terdiri 10 \nodes dengan total 80 prosesor. 

Pada \cluster ini sudah terpasang pustaka komputasi paralel OpenMPI\footnote{https://www.open-mpi.org/} dan MPICH\footnote{https://www.mpich.org/} serta paket dinamika molekular AMBER. Program paralel MPI dan eksperimen AMBER dijalankan mekanisme antrian \f{batch-jobs} untuk menjamin ketersediaan sumberdaya komputasi (\f{computing nodes}) saat program dieksekusi. Pengaturan eksekusi program paralel ini ditangani oleh \f{Sun Grid Engine}\footnote{http://www.rocksclusters.org/roll-documentation/sge/5.4/} yang juga tersedia dalam paket Rocks.
