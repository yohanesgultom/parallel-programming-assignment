%-----------------------------------------------------------------------------%
\chapter{\topikDua}
%-----------------------------------------------------------------------------%

%-----------------------------------------------------------------------------%
\section{Eksperimen}
%-----------------------------------------------------------------------------%
Pada bagian ini kami melakukan eksperimen perkalian matriks-vektor dan matriks bujursangkar pada CPU (sekuensial), CUDA, CUDA dengan \textit{shared memory}, CUBLAS dan CPU (\textit{cluster} MPI) untuk membandingkan kinerjanya.

\subsection{Perbandingan Program Sekuensial dan CUDA} 

\todo{}

\subsubsection{Deskripsi Program}

\todo{}

\subsubsection{Hasil Eksperimen}

\todo{}

%\begin{figure}
%	\centering
%	\includegraphics[width=1\textwidth]
%	{pics/process_topologies_demo}
%	\caption{Contoh eksekusi program demo proses topologi}
%	\label{fig:process_topologies_demo}
%\end{figure}  
%
%\begin{figure}
%	\centering
%	\includegraphics[width=1\textwidth]
%	{pics/process_topologies_creation_nbcr}
%	\caption{Eksperimen waktu pembuatan topologi kartesian di cluster UCSD}
%	\label{fig:process_topologies_creation_nbcr}
%\end{figure}  

\subsection{Program CUDA dengan Variasi ukuran \textit{Grid}/\textit{Block}} 

\todo{}

\subsubsection{Deskripsi Program}

\todo{}

\subsubsection{Hasil Eksperimen}

\todo{}

\subsection{Program CUDA \textit{Shared Memory}, CUBLAS dan MPI} 

\todo{}

\subsubsection{Deskripsi Program}

\todo{}

\subsubsection{Hasil Eksperimen}

\todo{}


%-----------------------------------------------------------------------------%
\section{Kesimpulan}
%-----------------------------------------------------------------------------%

\todo{}