%
% Template Laporan Skripsi/Thesis 
%
% @author  Andreas Febrian, Lia Sadita 
% @version 1.03
%
% Dokumen ini dibuat berdasarkan standar IEEE dalam membuat class untuk 
% LaTeX dan konfigurasi LaTeX yang digunakan Fahrurrozi Rahman ketika 
% membuat laporan skripsi. Konfigurasi yang lama telah disesuaikan dengan 
% aturan penulisan thesis yang dikeluarkan UI pada tahun 2008.
%

%
% Tipe dokumen adalah report dengan satu kolom. 
%
\documentclass[12pt, a4paper, onecolumn, oneside, final]{report}

% Load konfigurasi LaTeX untuk tipe laporan thesis
\usepackage{uithesis}


% Load konfigurasi khusus untuk laporan yang sedang dibuat
%-----------------------------------------------------------------------------%
% Informasi Mengenai Dokumen
%-----------------------------------------------------------------------------%
% 
% Judul laporan. 
\var{\judul}{PR 2 Kajian Bahasa Pemrograman Paralel}
% 
% Tulis kembali judul laporan, kali ini akan diubah menjadi huruf kapital
\Var{\Judul}{PR 2 Kajian Bahasa Pemrograman Paralel}
% 
% Tulis kembali judul laporan namun dengan bahasa Ingris
\var{\judulInggris}{PR 2 Kajian Bahasa Pemrograman Paralel}

% 
% Tipe laporan, dapat berisi Skripsi, Tugas Akhir, Thesis, atau Disertasi
\var{\type}{Laporan Tugas Pemrograman Paralel}
% 
% Tulis kembali tipe laporan, kali ini akan diubah menjadi huruf kapital
\Var{\Type}{Laporan Tugas Pemrograman Paralel}
% 
% Tulis nama penulis 
\var{\penulis}{Kelompok III}
% 
% Tulis kembali nama penulis, kali ini akan diubah menjadi huruf kapital
\Var{\Penulis}{Kelompok III}
% 
% Tulis NPM penulis
\var{\npm}{XXXXXXXXXX}

\var{\npmPertama}{1506706276}
\var{\npmKedua}{1506706295}
\var{\npmKetiga}{1506706345}

\var{\namaPertama}{Muhammad Fathurachman}
\var{\namaKedua}{Otniel Yosi Viktorisa}
\var{\namaKetiga}{Yohanes Gultom}

% 
% Tuliskan Fakultas dimana penulis berada
\Var{\Fakultas}{Ilmu Komputer}
\var{\fakultas}{Ilmu Komputer}
% 
% Tuliskan Program Studi yang diambil penulis
\Var{\Program}{Magister Ilmu Komputer}
\var{\program}{Magister Ilmu Komputer}
% 
% Tuliskan tahun publikasi laporan
\Var{\bulanTahun}{Mei 2016}
% 
% Tuliskan gelar yang akan diperoleh dengan menyerahkan laporan ini
%\var{\gelar}{Sarjana ??}
% 
% Tuliskan tanggal pengesahan laporan, waktu dimana laporan diserahkan ke 
% penguji/sekretariat
%\var{\tanggalPengesahan}{XX Januari 2010} 
% 
% Tuliskan tanggal keputusan sidang dikeluarkan dan penulis dinyatakan 
% lulus/tidak lulus
%\var{\tanggalLulus}{XX Januari 2010}
% 
% Tuliskan pembimbing 
\var{\pembimbing}{Prof. Heru Suhartanto}
% 
% Alias untuk memudahkan alur penulisan paa saat menulis laporan
\var{\saya}{Penulis}

%-----------------------------------------------------------------------------%
% Judul Setiap Bab
%-----------------------------------------------------------------------------%
% 
% Berikut ada judul-judul setiap bab. 
% Silahkan diubah sesuai dengan kebutuhan. 
% 
\Var{\lingkungan}{Lingkungan Percobaan}
\Var{\topikSatu}{R}
\Var{\topikDua}{OpenCL}
\Var{\kontribusi}{Kontribusi}

% Daftar pemenggalan suku kata dan istilah dalam LaTeX
\include{hype.indonesia}
% Daftar istilah yang mungkin perlu ditandai 
%
% @author  Andreas Febrian
% @version 1.00
% 
% Mendaftar seluruh istilah yang mungkin akan perlu dijadikan 
% italic atau bold pada setiap kemunculannya dalam dokumen. 
% 

\var{\license}{\f{Creative Common License 1.0 Generic}}
\var{\bslash}{$\setminus$}

\var{\cluster}{\f{cluster }}
\var{\Cluster}{\f{Cluster }}
\var{\multicores}{\f{multi-cores }}
\var{\node}{\f{node }}
\var{\nodes}{\f{nodes }}
\var{\Node}{\f{Node }}
\var{\Nodes}{\f{Nodes }}
\var{\manager}{\f{manager }}
\var{\Manager}{\f{Manager }}
\var{\worker}{\f{worker }}
\var{\Worker}{\f{Worker }}
\var{\speedup}{\f{speed-up }}
\var{\Speedup}{\f{Speed-up }}

% Awal bagian penulisan laporan
\begin{document}
%
% Sampul Laporan
\include{sampul}

%
% Gunakan penomeran romawi
\pagenumbering{roman}

%
% load halaman judul dalam
%\addChapter{HALAMAN JUDUL}
%\include{judul_dalam}

%
% setelah bagian ini, halaman dihitung sebagai halaman ke 2
\setcounter{page}{2}

%
% load halaman pengesahan
%\addChapter{LEMBAR PERSETUJUAN}
%\include{pengesahan}
%
% load halaman orisinalitas 
%\addChapter{LEMBAR PERNYATAAN ORISINALITAS}
%\include{orisinal}
%
%
%\addChapter{LEMBAR PENGESAHAN}
%\include{pengesahan_sidang}
%
%
%\addChapter{\kataPengantar}
%\include{pengantar}
%
%
%\addChapter{LEMBAR PERSETUJUAN PUBLIKASI ILMIAH}
%\include{persetujuan_publikasi}
%
% 
%\addChapter{ABSTRAK}
%\include{abstrak}
%
%
%\include{abstract}

%
% Daftar isi, gambar, dan tabel
%
\tableofcontents
\clearpage
\listoffigures
\clearpage
%\listoftables
%\clearpage

%
% Gunakan penomeran Arab (1, 2, 3, ...) setelah bagian ini.
%
\pagenumbering{arabic}

%
%
%
%-----------------------------------------------------------------------------%
\chapter{\lingkungan}
%-----------------------------------------------------------------------------%

\section{Lingkungan Pengembangan}

Program paralel yang digunakan di dalam eksperimen ini dibuat menggunakan bahasa C yang di-\f{compile} menggunakan pustaka OpenMPI pada sistem operasi Linux Ubuntu dan Mint.

Kode program-program dan laporan eksperimen ini kami simpan menggunakan layanan GitHub di alamat \f{https://github.com/yohanesgultom/parallel-programming-assignment}. Hal ini kami lakukan untuk mempermudah kolaborasi dalam pembuatan program dan laporan.

\section{Lingkungan Percobaan}

Dalam eksperimen yang dilakukan, kelompok kami menggunakan mesin-mesin dengan GPU NVIDIA dan \cluster CPU Rocks University of California San Diego (UCSD).

%-----------------------------------------------------------------------------%
\subsection{Server GPU Fasilkom Universitas Indonesia (UI)}
%-----------------------------------------------------------------------------%
\textit{Server} dengan GPU NVIDIA merupakan lingkungan utama percobaan ini. Kelompok kami melakukan percobaan di 2 buah server Fasilkom UI:

\begin{enumerate}
	\item Server GTX 980 (152.118.31.27)
	\begin{itemize}
		\item NVIDIA GTX 980 2048 CUDA Cores 4 GB GRAM
		\item Intel(R) Core(TM) i7-3770 CPU 4 cores @ 3.40GHz 
		\item RAM 2x8 GB DDR3 1600 Mhz
		\item SSD SAMSUNG MZ7TD128 128 GB		
		\item OS Debian 7 Wheezy x64
		\item CUDA 7.0
		\item Amber 14 dan Amber Tools 15
	\end{itemize}
	\item Server GTX 970 (152.118.31.34)
	\begin{itemize}
		\item NVIDIA GTX 970 1664 CUDA Cores 4 GB GRAM
		\item Intel(R) Core(TM) i7-3770 CPU 4 cores @ 3.40GHz
		\item RAM 2x8 GB DDR3 1600 Mhz
		\item SSD SAMSUNG MZ7TD128 128 GB		
		\item OS Debian 7 Wheezy x64
		\item CUDA 7.0
		\item Amber 14 dan Amber Tools 15
	\end{itemize}
\end{enumerate}

Semua \textit{server} ini dapat diakses dari jaringan Fasilkom menggunakan protokol \textit{SSH} dan \textit{credential Single Sign On} (SSO) UI. Sedangkan dari luar jaringan UI, semua \textit{server} ini dapat diakses dengan masuk lebih dahulu ke \url{kawung.cs.ui.ac.id} menggunakan protokol SSH juga.

%-----------------------------------------------------------------------------%
\subsection{Personal Computer (Laptop)}
%-----------------------------------------------------------------------------%

Sebagai bahan perbandingan, kami juga menggunakan PC (\textit{notebook}) yang juga menggunakan GPU NVIDIA yang mendukung CUDA:
\begin{itemize}
	\item NVIDIA 940M 384 CUDA Cores 2 GB GRAM
	\item Intel(R) Core(TM) i7-5500U CPU 2 cores @ 3.40GHz
	\item RAM 8 GB DDR3 1600 Mhz
	\item SSD Crucial 250 GB		
	\item OS Ubuntu 15.10 x64
	\item CUDA 7.5
	\item Amber 14 dan Amber Tools 15
\end{itemize}

%-----------------------------------------------------------------------------%
\subsection{Cluster Rocks University of California San Diego (UCSD)}
%-----------------------------------------------------------------------------%
\Cluster Rocks\footnote{www.rocksclusters.org} milik University of California San Diego (UCSD) ini dapat diakses pada alamat \f{nbcr-233.ucsd.edu} menggunakan protokol SSH dari komputer yang telah didaftarkan \f{public key} nya. Berdasarkan informasi dari aplikasi \f{monitoring} Ganglia \footnote{http://nbcr-233.ucsd.edu/ganglia}, cluster ini terdiri 10 \nodes dengan total 80 prosesor. 

Pada \cluster ini sudah terpasang pustaka komputasi paralel OpenMPI\footnote{https://www.open-mpi.org/} dan MPICH\footnote{https://www.mpich.org/} serta paket dinamika molekular AMBER. Program paralel MPI dan eksperimen AMBER dijalankan mekanisme antrian \f{batch-jobs} untuk menjamin ketersediaan sumberdaya komputasi (\f{computing nodes}) saat program dieksekusi. Pengaturan eksekusi program paralel ini ditangani oleh \f{Sun Grid Engine}\footnote{http://www.rocksclusters.org/roll-documentation/sge/5.4/} yang juga tersedia dalam paket Rocks.

%-----------------------------------------------------------------------------%
\chapter{\topikSatu}
%-----------------------------------------------------------------------------%

%-----------------------------------------------------------------------------%
\section{Pendahuluan}
%-----------------------------------------------------------------------------%
Topik eksperimen pertama adalah perkalian matriks-vektor dan perkalian matriks bujur sangkar dengan beberbagai algoritma paralel. 

\subsection{Perkalian Matriks-Vektor} 

\subsubsection{Row-Wise Decomposition}

Algoritma paralel perkalian matriks-vektor yang paling sederhana, yaitu memecah proses perkalian berdasarkan baris matriks (\f{row-wise}). Setiap prosesor akan bertanggung jawab untuk mengalikan sebuah baris matriks dan vektor pada satu waktu. Jika jumlah prosesor ($np$) lebih sedikit dari jumlah baris matriks ($r$) maka setiap prosesor bertugas mengalikan $n = \frac{r}{np}$ secara sekuensial.

\begin{figure}
	\centering
	\includegraphics[width=0.75\textwidth]
	{pics/mv_rowwise}
	\caption{Perkalian matriks-vektor Row-Wise Decomposition}
	\label{fig:mv_rowwise}
\end{figure}  

\subsubsection{Column-wise Decomposition}

Algoritma perkalian matriks-vektor ini merupakan alternatif dari \f{row-wise decomposition} di mana pemecahan proses perkalian dilakukan berdasarkan kolom matriks. Setiap proses akan mengalikan sebuah kolom matriks dan sebuah elemen vektor pada satu waktu. Mirip dengan algoritma \f{row-wise decomposition}, jika jumlah prosesor ($np$) lebih sedikit dari jumlah kolom matriks ($c$) maka setiap prosesor bertugas mengalikan $n = \frac{c}{np}$ secara sekuensial.

\begin{figure}
	\centering
	\includegraphics[width=0.9\textwidth]
	{pics/mv_colwise}
	\caption{Perkalian matriks-vektor Column-Wise Decomposition}
	\label{fig:mv_colwise}
\end{figure}  

\subsubsection{Checkerboard Decomposition}

Algoritma perkalian matriks-vektor \f{checkerboard decomposition} ini membagi matriks menjadi submatriks dengan ukuran yang sama dan mengalikannya dengan subvektor yang sesuai. Hasil perkalian tersebut kemudian akan dijumlahkan dan dipetakan ke vektor hasil.

Prekondisi dari algoritma ini adalah jumlah elemen matriks ($n$) harus bisa dibagi rata ke sejumlah prosesor ($np$) atau dengan kata lain memenuhi persamaan \ref{eq:mv_checkerboard}. Hasil pembagian ini ($x$) akan menjadi ukuran submatriks (dan subvektor) yang dikerjakan di tiap proses secara paralel.

\begin{equation}
	x = \sqrt{\frac{n}{np}},\text{ where } x \in \mathbb{Z}
	\label{eq:mv_checkerboard}
\end{equation}


\begin{figure}
	\centering
	\includegraphics[width=0.75\textwidth]
	{pics/mv_checkerboard}
	\caption{Perkalian matriks-vektor Checkerboard Decomposition}
	\label{fig:mv_checkerboard}
\end{figure}  

\subsection{Perkalian Matriks Bujursangkar} 

\subsubsection{Row-Wise Decomposition}

Mirip dengan algoritma perkalian matriks-vektor \f{row-wise decomposition}, algortima ini juga membagi pekerjaan berdasarkan baris dari matriks pertama seperti yang diilustrasikan pada gambar \ref{fig:mm_rowwise}. Setiap prosesor akan mengalikan sebuah baris dari matriks pertama $A$ dengan seluruh kolom dari matriks kedua $B$. Hasil dari seluruh prosesor kemudian dikonkatenasi menjadi matriks baru $C$.

\begin{figure}
	\centering
	\includegraphics[width=1\textwidth]
	{pics/mm_rowwise}
	\caption{Perkalian matriks bujursangkar Row-Wise Decomposition}
	\label{fig:mm_rowwise}
\end{figure}

\subsubsection{Cannon}

Algoritma Cannon menggunakan dekomposisi seperti algortima matriks-vektor \f{checkerboard decomposition} di mana matriks $A$ dan $B$ dibagi menjadi menjadi submatriks bujursangkar. Perbedaan pada algoritma Cannon adalah prekondisi di mana jumlah proses ($np$) harus merupakan bujursangkar sempurna (\f{perfect square}). Tujuan utama dari algoritma ini adalah untuk meningkatkan efisiensi penggunaan memori pada proses paralel.

\begin{figure}
	\centering
	\includegraphics[width=1\textwidth]
	{pics/mm_cannon}
	\caption{Perkalian matriks bujursangkar Cannon}
	\label{fig:mm_cannon}
\end{figure}

\subsubsection{Fox}

Algoritma Fox memiliki kemiripan dengan algoritma Cannon dalam hal dekomposisi matriks menjadi submatriks bujursangkar dan pemetaan prosesor yang harus dapat membentuk bujursangkar sempurna (\f{perfect square}). Yang membedakan algoritma Fox dan Cannon adalah skema distribusi awal submatriks ke prosesor

\begin{figure}
	\centering
	\includegraphics[width=0.75\textwidth]
	{pics/mm_fox}
	\caption{Perkalian matriks bujursangkar Fox}
	\label{fig:mm_fox}
\end{figure}

\subsubsection{DNS}

Algoritma yang diberi nama berdasarkan nama pembuatnya (Dekel, Nassimi and Aahni) ini, diajukan dalam rangka meningkatkan lagi efisiensi penggunaan memori pada perkalian matriks bujursangkar secara paralel. Karakteristik algoritma ini adalah:
\begin{itemize}
	\item Berdasarkan partisi \f{intermediate data}
	\item Melakukan perkalian skalar $n^3$ sehingga membutuhkan proses sebanyak $n \times n \times n$
	\item Membutuhkan waktu komputasi $O(\log n)$ dengan menggunakan $O(\frac{n^3}{\log n})$
\end{itemize}

\begin{figure}
	\centering
	\includegraphics[width=1\textwidth]
	{pics/mm_dns1}
	\caption{Perkalian matriks bujursangkar DNS iterasi 1}
	\label{fig:mm_dns1}
\end{figure}

\begin{figure}
	\centering
	\includegraphics[width=1\textwidth]
	{pics/mm_dns2}
	\caption{Perkalian matriks bujursangkar DNS iterasi 2}
	\label{fig:mm_dns2}
\end{figure}


%-----------------------------------------------------------------------------%
\section{Eksperimen}
%-----------------------------------------------------------------------------%

\subsection{Perkalian Matriks-Vektor} 

\subsubsection{Row-Wise Decomposition}

Deskripsi program yang digunakan:
\begin{itemize}
	\item Sumber kode: \texttt{\url{ https://github.com/yohanesgultom/parallel-programming-assignment/blob/master/problem1/mv_rowwise.c}}
	\item Satu prosesor berlaku sebagai \manager dan sisanya berperan sebagai \worker 
	\item Tugas \manager adalah menginisialisasi matriks dan vektor, mendistribusikannya secara \f{row-wise decomposition} menggunakan \verb|MPI_Send| dan \verb|MPI_Bcast| dan mengumpulkan hasil dari tiap \worker dengan \verb|MPI_Recv|	
	\item Waktu eksekusi dan komunikasi dihitung (dalam detik) menggunakan \verb|MPI_Wtime|
\end{itemize}

Eksperimen dilakukan di \cluster UCSD dan Fasilkom dengan variasi jumlah prosesor di mana waktu yang diukur adalah nilai rata-rata dari lima kali eksekusi.

\begin{figure}
	\centering
	\includegraphics[width=1\textwidth]
	{pics/chart_mv_rowwise_nbcr}
	\caption{Grafik hasil eksperimen Row-Wise Decomposition Cluster UCSD}
	\label{fig:result_mv_rowwise_nbcr}
\end{figure}  

\begin{figure}
	\centering
	\includegraphics[width=1\textwidth]
	{pics/chart_mv_rowwise_fasilkom}
	\caption{Grafik hasil eksperimen Row-Wise Decomposition Cluster Fasilkom UI}
	\label{fig:result_mv_rowwise_fasilkom}
\end{figure}  

Pada grafik gambar \ref{fig:result_mv_rowwise_nbcr}, terlihat bahwa pada \cluster UCSD terjadi penurunan waktu eksekusi/komunikasi (\speedup) ketika digunakan 3, 4, 5, 7 prosesor (2, 3, 4, 6 \worker). Tapi ketika digunakan 9 prosesor (8 \worker), waktu yang dibutuhkan malah meningkat (tidak terjadi \speedup). Kami tidak mencoba lebih dari 9 prosesor karena sepertinya hasilnya akan sama (tidak ada \speedup).

Sedangkan pada \cluster Fasilkom UI, kami hanya mencoba 3, 5, 7 prosesor karena masalah \f{permission} yang mengakibatkan hanya satu \node (8 prosesor) yang bisa digunakan. Seperti yang terlihat pada grafik gambar \ref{fig:result_mv_rowwise_fasilkom}, \speedup hanya terjadi pada prosesor 3-5. Sedangkan dari 5-7 prosesor tidak terjadi \speedup (waktu ekskusi hampir sama).

Kami tidak melakukan pengujian pada \cluster Rocks karena keterbatasan jumlah prosesor pada \f{laptop} yang kami gunakan.

\subsubsection{Column-wise Decomposition}

Deskripsi program yang digunakan:
\begin{itemize}
	\item Sumber kode: \texttt{\url{ https://github.com/yohanesgultom/parallel-programming-assignment/blob/master/problem1/mv_columnwise.c}}
	\item Satu prosesor berlaku sebagai \manager dan sisanya berperan sebagai \worker 
	\item Tugas \manager adalah menginisialisasi matriks dan vektor, mendistribusikannya secara \f{column-wise decomposition} menggunakan \verb|MPI_Send| dan \verb|MPI_Bcast| dan mengumpulkan hasil dari tiap \worker dengan \verb|MPI_Recv|	
	\item Waktu eksekusi dan komunikasi dihitung (dalam detik) menggunakan \verb|MPI_Wtime|
\end{itemize}

Eksperimen dilakukan di \cluster UCSD dan Fasilkom dengan variasi jumlah prosesor di mana waktu yang diukur adalah nilai rata-rata dari lima kali eksekusi.

\begin{figure}
	\centering
	\includegraphics[width=1\textwidth]
	{pics/chart_mv_colwise_nbcr}
	\caption{Grafik hasil eksperimen Column-Wise Decomposition Cluster UCSD}
	\label{fig:result_mv_colwise_nbcr}
\end{figure}  

\begin{figure}
	\centering
	\includegraphics[width=1\textwidth]
	{pics/chart_mv_colwise_fasilkom}
	\caption{Grafik hasil eksperimen Column-Wise Decomposition Cluster Fasilkom UI}
	\label{fig:result_mv_colwise_fasilkom}
\end{figure}  

Pada grafik gambar \ref{fig:result_mv_colwise_nbcr}, terlihat bahwa pada \cluster UCSD terjadi penurunan waktu eksekusi/komunikasi (\speedup) ketika digunakan 4-5 prosesor. Tapi ketika digunakan 3-4, 5-9 prosesor, waktu yang dibutuhkan malah meningkat (tidak terjadi \speedup).

Sedangkan pada \cluster Fasilkom UI, kami hanya mencoba 3, 5, 7 prosesor karena masalah \f{permission} yang mengakibatkan hanya satu \node (8 prosesor) yang bisa digunakan. Seperti yang terlihat pada grafik gambar \ref{fig:result_mv_colwise_fasilkom}, \speedup hanya terjadi pada prosesor 3-5. Sedangkan dari 5-7 prosesor tidak terjadi \speedup (waktu ekskusi meningkat).

Kami tidak melakukan pengujian pada \cluster Rocks karena keterbatasan jumlah prosesor pada \f{laptop} yang kami gunakan.

\subsubsection{Checkerboard Decomposition}

Deskripsi program yang digunakan:
\begin{itemize}
	\item Sumber kode: \texttt{\url{ https://github.com/yohanesgultom/parallel-programming-assignment/blob/master/problem1/mv_checkerboard.c}}
	\item Satu prosesor berlaku sebagai \manager dan sisanya berperan sebagai \worker 
	\item Tugas \manager adalah menginisialisasi matriks dan vektor, mendistribusikannya secara \f{checkerboard decomposition} menggunakan \verb|MPI_Send| dan \verb|MPI_Bcast| dan mengumpulkan hasil dari tiap \worker dengan \verb|MPI_Recv|	
	\item Waktu eksekusi dan komunikasi dihitung (dalam detik) menggunakan \verb|MPI_Wtime|
\end{itemize}

Eksperimen dilakukan di \cluster UCSD dan Fasilkom dengan variasi jumlah prosesor di mana waktu yang diukur adalah nilai rata-rata dari lima kali eksekusi.

\begin{figure}
	\centering
	\includegraphics[width=1\textwidth]
	{pics/chart_mv_checkerboard_nbcr}
	\caption{Grafik hasil eksperimen Checkerboard Cluster UCSD}
	\label{fig:result_mv_checkerboard_nbcr}
\end{figure}  

\begin{figure}
	\centering
	\includegraphics[width=1\textwidth]
	{pics/chart_mv_checkerboard_fasilkom}
	\caption{Grafik hasil eksperimen Checkerboard Cluster Fasilkom UI}
	\label{fig:result_mv_checkerboard_fasilkom}
\end{figure}  

Pada grafik gambar \ref{fig:result_mv_checkerboard_nbcr}, terlihat bahwa waktu yang dibutuhkan malah meningkat (tidak terjadi \speedup) ketika kami meningkatkan jumlah prosesor dari 5-7. Kami tidak mencoba prosesor yang lebih banyak karena melihat pola yang sama (tidak ada \speedup).

Sedangkan pada \cluster Fasilkom UI, kami hanya mencoba 5 prosesor karena masalah \f{permission} yang mengakibatkan hanya satu \node (8 prosesor) yang bisa digunakan. Seperti yang terlihat pada grafik gambar \ref{fig:result_mv_checkerboard_fasilkom}, kami hanya mencoba melakukan eksekusi dengan ukuran matriks/vektor yang berbeda (360x360 dan 1152x1152).

Kami tidak melakukan pengujian pada \cluster Rocks karena keterbatasan jumlah prosesor pada \f{laptop} yang kami gunakan.

\subsection{Perkalian Matriks Bujursangkar} 

\subsubsection{Row-Wise Decomposition}

\subsubsection{Cannon}

\subsubsection{Fox}

\subsubsection{DNS}






%-----------------------------------------------------------------------------%
\chapter{\topikDua}
%-----------------------------------------------------------------------------%

%-----------------------------------------------------------------------------%
\section{Pendahuluan}
%-----------------------------------------------------------------------------%

OpenCL \textit{Open Computing Language} merupakan \textit{library General Purpose Graphics Processing Unit} Computing (GPGPU) yang dikembangkan oleh Khronos (yang disponsori oleh Apple). OpenCL juga disebut sebagai sebuah \textit{open standard} untuk pemrograman paralel pada sistem heterogen karena mendukung berbagai vendor GPU (\textit{integrated} maupun \textit{dedicated}) seperti Intel, AMD, NVIDIA, Apple dan ARM.

OpenCL merupakan \textit{library} yang dapat berjalan di kebanyakan sistem karena \textit{kernel} bahasanya merupakan subset dari C++ 14. Selain itu, OpenCL juga telah memiliki \textit{language binding} dari bahasa pemrograman \textit{high-level} seperti Microsoft.Net (NOpenCL dan OpenCL.Net), Erlang dan Python (PyOpenCL).

OpenCL saat ini sudah mencapai versi 2.0 dengan sejarah pengembangan \cite{ opencl.mukherjee} sebagai berikut:

\begin{itemize}
	\item OpenCL 1.0​
	\begin{itemize}
		\item Model pemrograman dasar
	\end{itemize}
	\item OpenCL 1.1 and 1.2​
	\begin{itemize}
		\item Teknik manajemen \textit{memory}
		\item Kontrol \textit{resources} yang lebih baik
	\end{itemize}
	\item OpenCL 2.0​
	\begin{itemize}
		\item Memaksimalkan penggunaan kapabilitas baru \textit{hardware}
		\item API pemrograman yang lebih baik
		\item Kontrol \textit{resources} yang lebih baik
	\end{itemize}
\end{itemize}

%-----------------------------------------------------------------------------%
\subsection{Instalasi}
%-----------------------------------------------------------------------------%

Salah satu kelebihan yang dimiliki OpenCL dibanding \textit{hardware-specific library} seperti NVIDIA CUDA adalah dukungan ke banyak vendor \textit{hardware}. Untuk mencapai hal ini, OpenCL beradaptasi dengan karakteristik instalasi masing-masing vendor sehingga setiap vendor memiliki prosedur instalasi OpenCL yang berbeda. Berikut daftar tautan panduan instalasi untuk beberapa vendor ternama:

\begin{itemize}
	\item AMD \url{http://developer.amd.com/tools-and-sdks/opencl-zone​}
	\item Intel \url{https://software.intel.com/en-us/intel-opencl​}
	\item NVIDIA \url{https://developer.nvidia.com/opencl}
\end{itemize}

Contoh langkah-langkah instalasi OpenCL SDK pada Ubuntu 15.10 64-bit dengan NVIDIA 940M \cite{opencl.howto} adalah sebagai berikut:

\begin{enumerate}
	\item Instal \textit{driver} yang disarankan oleh versi Ubuntu 15.10 yaitu NVIDIA \textit{driver} versi 352. Instalasi dapat dilakukan melalui menu \textit{Additional Drivers} atau dengan mengetikkan perintah pada terminal:
	
	\begin{lstlisting}
		
	$ sudo apt-get install nvidia-352
	\end{lstlisting}
	
	\item Setelah itu, instal CUDA dengan mengunduh \textit{repository} CUDA Toolkit versi 7.5 untuk Ubuntu 15.04 (*.deb) di \url{https://developer.nvidia.com/cuda-downloads} dan menjalankan perintah berikut di \textit{terminal}:  
	
	\begin{lstlisting}
	
		$ sudo dpkg -i cuda-repo-ubuntu1504-7-5-*_amd64.deb
		$ sudo apt-get update
		$ sudo apt-get install cuda-toolkit
	\end{lstlisting}
	Pastikan juga baris-baris ini ada di dalam file \verb|~/.bashrc| (baris terbawah):
	
	\begin{lstlisting}
	
		export CUDA_HOME=/usr/local/cuda-7.5 
		export LD_LIBRARY_PATH=${CUDA_HOME}/lib64 
		PATH=${CUDA_HOME}/bin:${PATH} 
		export PATH	
	\end{lstlisting}
	
	Untuk memastikan bahwa driver dan CUDA sudah terinstal sempurna nama GPU (contoh NVIDIA 940M) harus terlihat ketika 2 kelompok perintah ini dipanggil:
	
	Pertama:
	
	\begin{lstlisting}
	
		$ nvidia-smi
	\end{lstlisting}
	
	Kedua:
	
	\begin{lstlisting}
	
		$ cd $CUDA_HOME/samples/1_Utilities/deviceQuery
		$ sudo make run		
	\end{lstlisting}
	
	\item Terakhir, instal header OpenCL dengan perintah:
	
	\begin{lstlisting}
	
	$ sudo apt-get install nvidia-352-dev nvidia-prime nvidia-modprobe nvidia-opencl-dev
	\end{lstlisting}
	
\end{enumerate}

%-----------------------------------------------------------------------------%
\subsection{Struktur Program}
%-----------------------------------------------------------------------------%

Struktur program OpenCL cukup berbeda dengan struktur program CUDA. Perbedaan mendasar adalah adanya proses kompilasi kernel di dalam program OpenCL di mana untuk CUDA proses tersebut tidak perlu dilakukan secara eksplisit pada program. Oleh karena itu, kernel pada program OpenCL biasanya diletakkan di file terpisah dengan ekstensi \verb|*.cl|.

Struktur umum atau langkah-langkah pada program OpenCL adalah sebagai berikut:

\begin{enumerate}
		\item Memilih \textit{platform} yang tersedia
		\item Memilih \textit{device} pada \textit{platform} yang tersedia
		\item Membuat \textit{Context} ​
		\item Membuat \textit{command queue}
		\item Membuat \textit{memory objects} ​
		\item Membaca file \textit{kernel}
		\item Membuat \textit{program object}
		\item Mengkompilasi \textit{kernel}
		\item Membuat \textit{kernel object}
		\item Memasukkan \textit{kernel arguments}
		\item Menjalankan \textit{kernel}
		\item Membaca \textit{memory object} (hasil proses \textit{kernel})
		\item \textit{Free memory objects}
\end{enumerate}

Contoh program sederhana OpenCL \textit{Single-Precision A·X Plus Y} (SAXPY) dapat dilihat pada tautan berikut:

\begin{itemize}
	\item Program utama \url{https://github.com/yohanesgultom/parallel-programming-assignment/blob/master/PR2/opencl/saxpy.c}
	\item Kernel \url{https://github.com/yohanesgultom/parallel-programming-assignment/blob/master/PR2/opencl/saxpy.cl}
\end{itemize}

%-----------------------------------------------------------------------------%
\subsection{Perbandingan Terminologi dengan CUDA}
%-----------------------------------------------------------------------------%

Bagi \textit{programmer} yang sudah terbiasa dengan CUDA, bada bagian ini akan dipaparkan tabel-tabel konversi terminologi antara CUDA dan OpenCL \cite{opencl.cuda.porting}. Dengan tabel-table ini diharapkan \textit{programmer} CUDA dapat lebih cepat memahami OpenCL dan mengkonversi program CUDA ke OpenCL. 

\begin{table}
	\centering
	\caption{Terminologi Perangkat Keras}
	\label{tab:terminologi_perangkat_keras}
	\begin{tabular}{|C{6cm}|C{6cm}|}
		\rowcolor[gray]{.9} \hline \rule[-2ex]{0pt}{5.5ex} CUDA & OpenCL \\ 
		\hline \rule[-2ex]{0pt}{5.5ex} Stream Multiprocessor (SM) & CU (Compute Unit) \\ 
		\hline \rule[-2ex]{0pt}{5.5ex} Thread & Work-item \\ 
		\hline \rule[-2ex]{0pt}{5.5ex} Block & Work-group \\ 
		\hline \rule[-2ex]{0pt}{5.5ex} Global Memory & Global Memory \\ 
		\hline \rule[-2ex]{0pt}{5.5ex} Constant Memory & Constant Memory \\ 
		\hline \rule[-2ex]{0pt}{5.5ex} Shared Memory & Local Memory \\ 
		\hline \rule[-2ex]{0pt}{5.5ex} Local Memory & Private Memory \\ 
		\hline 
	\end{tabular} 
\end{table}

\begin{table}
	\centering
	\caption{Qualifiers untuk fungsi Kernel}
	\label{tab:qualifiers_untuk_fungsi_kernel}
	\begin{tabular}{|C{6cm}|C{6cm}|}
		\rowcolor[gray]{.9} \hline \rule[-2ex]{0pt}{5.5ex} CUDA & OpenCL \\ 
		\hline \rule[-2ex]{0pt}{5.5ex} \verb|_global__ function|​ & \verb|__kernel function| \\ 
		\hline \rule[-2ex]{0pt}{5.5ex} \verb|__device__ function| & N/A \\ 
		\hline \rule[-2ex]{0pt}{5.5ex} \verb|__constant__ variable|​ & \verb|__constant variable| \\ 
		\hline \rule[-2ex]{0pt}{5.5ex} \verb|__device__ variable| & \verb|__global variable| \\ 
		\hline \rule[-2ex]{0pt}{5.5ex} \verb|__shared__ variable| & \verb|__local variable| \\ 
		\hline 
	\end{tabular} 
\end{table}


\begin{table}
	\centering
	\caption{Indeks pada Kernel}
	\label{tab:indeks_pada_kernel}
	\begin{tabular}{|C{6cm}|C{6cm}|}
		\rowcolor[gray]{.9} \hline \rule[-2ex]{0pt}{5.5ex} CUDA & OpenCL \\ 
		\hline \rule[-2ex]{0pt}{5.5ex} \verb|gridDim|​ & \verb|get_num_groups()| \\ 
		\hline \rule[-2ex]{0pt}{5.5ex} \verb|blockDim| & \verb|get_local_size()​| \\ 
		\hline \rule[-2ex]{0pt}{5.5ex} \verb|blockIdx|​ & \verb|get_group_id()| \\ 
		\hline \rule[-2ex]{0pt}{5.5ex} \verb|threadIdx| & \verb|get_local_id()| \\ 
		\hline \rule[-2ex]{0pt}{5.5ex} \verb|blockIdx * blockDim| \newline \verb| + threadIdx| & \verb|get_global_id()| \\ 
		\hline \rule[-2ex]{0pt}{5.5ex} \verb|gridDim * blockDim| & \verb|get_global_size()| \\ 		
		\hline 
	\end{tabular} 
\end{table}

\begin{table}
	\centering
	\caption{Pemanggilan API}
	\label{tab:pemanggilan_api}
	\begin{tabular}{|C{6cm}|C{6cm}|}
		\rowcolor[gray]{.9} \hline \rule[-2ex]{0pt}{5.5ex} CUDA & OpenCL \\ 
		\hline \rule[-2ex]{0pt}{5.5ex} \verb|cudaGetDeviceProperties()|​ & \verb|clGetDeviceInfo()| \\ 
		\hline \rule[-2ex]{0pt}{5.5ex} \verb|cudaMalloc()| & \verb|clCreateBuffer()| \\ 
		\hline \rule[-2ex]{0pt}{5.5ex} \verb|cudaMemcpy()|​ & \verb|clEnqueueReadBuffer()| \newline \verb|clEnqueueWriteBuffer()| \\ 
		\hline \rule[-2ex]{0pt}{5.5ex} \verb|cudaFree()| & \verb|clReleaseMemObj()| \\ 
		\hline \rule[-2ex]{0pt}{5.5ex} \verb|kernel<<<...>>>()| & \verb|clEnqueueNDRangeKernel()| \\ 
		\hline 
	\end{tabular} 
\end{table}

%-----------------------------------------------------------------------------%
\subsection{Library BLAS}
%-----------------------------------------------------------------------------%

\textit{Basic Linear Algebra Subprograms} (BLAS) adalah \textit{library} yang umum digunakan pada pemrograman paralel karena berisi subprogram perkalian matriks dan vektor dasar. BLAS \cite{blas.wiki} awalnya merupakan bagian dari \textit{library} Fortran tetapi kemudian dikembangkan secara terpisah dan terbuka untuk bahasa C dan bahasa lainnya. BLAS terdiri dari 3 tingkat atau kelompok subprogram \cite{blas.netlib}:

\begin{itemize}
	\item Level 1 BLAS: operasi skalar, vektor and vektor-vektor​
	\item Level 2 BLAS: operasi matriks-vektor​
	\item Level 3 BLAS: operasi matriks-matriks
\end{itemize}

OpenCL memiliki beberapa alternatif implementasi BLAS yang dikembangkan oleh beberapa pihak, yaitu:

\begin{enumerate}
	\item ClBLAS
	
	ClBLAS \cite{opencl.clblas} merupakan implementasi \textit{library} BLAS untuk OpenCL yang bersifat \textit{opensource} yang dikembangkan oleh clMath\footnote{https://github.com/clMathLibraries}. \textit{Library} ini sudah mengimplementasikan BLAS secara lengkap (\textit{level} 1, 2 dan 3) dan juga memiliki fitur optimasi khusus untuk AMD GPU. Sisten operasi yang didukung oleh \textit{library} ini adalah Windows ® 7/8​, Linux ​dan Mac OSX. Cara pemanggilan fungsi pada clBLAS dapat dilihat pada gambar \ref{fig:clblas_sgemm}.

	\begin{figure}
		\centering
		\includegraphics[width=1.0\textwidth]
		{pics/clblas_sgemm.png}
		\caption{Cara pemanggilan clblasSgemm pada clBLAS}
		\label{fig:clblas_sgemm}
	\end{figure}
	
	\item MyGEMM
	
	MyGEMM \cite{opencl.mygemm} adalah \textit{library} yang dikembangkan oleh Cedric Nugteren yang juga bersifat \textit{opensource}. Library ini dikembangkan karena ketidakpuasannya terhadap kinerja clBLAS \cite{opencl.clblas.tutorial} pada NVIDIA GPU. Sekalipun \textit{library} ini dapat dioptimasi untuk NVIDIA GPU, implementasi BLAS yang ada baru \textit{single-precision generalised matrix-multiplication} (SGEMM) saja. Cara pemakaiannya juga sama persis dengan memakai \textit{kernel} OpenCL pada umumnya karena semua fungsi-fungsi BLAS diimplementasikan dalam bentuk file \textit{kernel} OpenCL biasa.	
	
\end{enumerate}

Perbandingan kinerja clBLAS, MyGEMM dan CUBLAS (CUDA) \cite{opencl.clblas.tutorial} pada GPU NVIDIA Tesla K40 dapat dilihat pada gambar \ref{fig:opencl_blas_performance}. Pada grafik tersebut terlihat bahwa kinerja CUDA dengan \textit{library} CUBLAS jauh lebih baik dari semua \textit{library} OpenCL. Di sini Nugteren juga menunjukkan bahwa implentasi SGEMM pada myGEMM lebih baik dari clBLAS.

\begin{figure}
	\centering
	\includegraphics[width=1.0\textwidth]
	{pics/opencl_blas_performance.png}
	\caption{Kinerja library BLAS pada OpenCL}
	\label{fig:opencl_blas_performance}
\end{figure}


%-----------------------------------------------------------------------------%
\subsection{Eksperimen}
%-----------------------------------------------------------------------------%

\subsubsection{Device Query}

OpenCL menyediakan \textit{Application Programming Interface} (API) untuk mengecek \textit{platform} dan \textit{device} OpenCL yang ada di dalam sebuah sistem \cite{opencl.ebook} \cite{opencl.clgetplatforminfo}, yaitu:

\begin{itemize}
	\item \verb|clGetPlatformID()|: mendapatkan daftar \textit{platform} pada mesin
	\item \verb|clGetPlatfrmInfo()|: mendapatkan informasi dari suatu \textit{platform}
	\item \verb|clGetDeviceID()|: mendapatkan daftar \textit{device} pada \textit{platform}
	\item \verb|clGetDeviceInfo()|: mendapatkan informasi dari suatu \textit{device}
\end{itemize}

API ini juga digunakan pada saat menjalankan \textit{kernel} untuk memilih \textit{platform} dan \textit{device} yang ingin digunakan untuk menjalankan \textit{kernel} tersebut.

Pada eksperimen ini, program \verb|device_query.c|\footnote{\url{https://github.com/yohanesgultom/parallel-programming-assignment/blob/master/PR2/opencl/device_query.c}} akan memanggil API untuk mendapatkan informasi \textit{platform} dan \textit{device} kemudian menampilkannya seperti pada gambar \ref{fig:device_query}.

\begin{figure}
	\centering
	\includegraphics[width=1.0\textwidth]
	{pics/device_query.png}
	\caption{Program untuk mendapatkan informasi platform dan device OpenCL}
	\label{fig:device_query}
\end{figure}

\subsubsection{Single-Precision A·X Plus Y (SAXPY)}

\textit{Single-Precision A·X Plus Y} (SAXPY) adalah program yang melakukan kombinasi perkalian skalar dan penjumlahan vektor $z = \alpha x + y$ di mana $x, y, z$: vektor dan $\alpha$: skalar. Program ini merupakan contoh program yang sederhana tapi cukup merepresentasikan sintaks operasi aljabar linear dari sebuah bahasa program atau \textit{library} sehingga sering dianggap sebagai \textit{"hello world"} untuk program aljabar linear. 

Program SAXPY pada eksperimen ini terdiri dari dua buah file yaitu \verb|saxpy.c|\footnote{\url{https://github.com/yohanesgultom/parallel-programming-assignment/blob/master/PR2/opencl/saxpy.c}} (program utama) dan \verb|saxpy.cl|\footnote{\url{https://github.com/yohanesgultom/parallel-programming-assignment/blob/master/PR2/opencl/saxpy.cl}} (\textit{kernel}). Program ini akan menghitung SAXPY dengan vektor yang berukuran 1.024 elemen dan mencetak hasilnya seperti pada gambar \ref{fig:saxpy}.

\begin{figure}
	\centering
	\includegraphics[width=1.0\textwidth]
	{pics/saxpy.png}
	\caption{Program SAXPY 1024 elemen}
	\label{fig:saxpy}
\end{figure}

\subsubsection{Perkalian Matriks Bujursangkar}

Eksperimen ini mencoba membandingkan kinerja perkalian matriks bujursangkar OpenCL \cite{opencl.mmul} dengan CUDA. Program yang digunakan adalah \verb|mmul_cuda.cu|\footnote{\url{https://github.com/yohanesgultom/parallel-programming-assignment/blob/master/PR2/opencl/mmul_cuda.cu}} yang melakukan perkalian matriks bujursangkar dengan CUDA. Sedangkan perkalian matriks bujursangkan OpenCL terdiri dari dua file yaitu \verb|mmul_opencl.c|\footnote{\url{https://github.com/yohanesgultom/parallel-programming-assignment/blob/master/PR2/opencl/mmul_opencl.c}} (program utama) dan \verb|mmul_opencl.cl|\footnote{\url{https://github.com/yohanesgultom/parallel-programming-assignment/blob/master/PR2/opencl/mmul_opencl.c}} (\textit{kernel}).

Hasil eksperimen pada mesin dengan NVIDIA 940M, memberikan hasil seperti grafik pada gambar \ref{fig:mmul_opencl_cuda}. Pada grafik tersebut bahwa untuk ukuran matriks 256x256 sampai 4096x2096, program perkalian matriks dengan OpenCL selalu sedikit lebih lambat dari program yang dibuat dengan CUDA. Sekalipun demikian untuk kasus ini, perbedaan kecepatan antara OpenCL dan CUDA sangatlah kecil, yaitu di bawah 0.3 detik.

\begin{figure}
	\centering
	\includegraphics[width=1.0\textwidth]
	{pics/mmul_opencl_cuda.png}
	\caption{Perbandingan perkalian matriks bujursangkar OpenCL dengan CUDA}
	\label{fig:mmul_opencl_cuda}
\end{figure}

\subsubsection{Gaussian Filter Blurring pada Gambar Bitmap}

\textit{Gaussian filter blurring} adalah teknik untuk membuat gambar menjadi \textit{blur} (kabur) (seperti gambar \ref{fig:blur_result}) dengan memanfaatkan perkalian dengan matriks yang dibangun menggunakan persamaan Gaussian \ref{eq:gaussian}.

\noindent \begin{align}\label{eq:gaussian}
g(x,y) = \frac{1}{2\pi \sigma^2} \cdot e^{-\frac{x^2 + y^2}{2 \sigma^2}}
\end{align}

\begin{figure}
	\centering
	\includegraphics[width=0.5\textwidth]
	{pics/blur_result.png}
	\caption{Proses Gaussian Filter Blurring pada gambar}
	\label{fig:blur_result}
\end{figure}

Eksperimen ini mencoba membandingkan kinerja Gaussian \textit{bluring} dengan CPU dan dengan GPU (menggunakan OpenCL) \cite{opencl.gaussianblur}. Program yang digunakan terdiri dari kumpulan \textit{library file}\footnote{\url{https://github.com/yohanesgultom/parallel-programming-assignment/tree/master/PR2/opencl/OpenCL_Gaussian_Blur}} dengan file utama \verb|main.c| (program utama) dan \verb|kernel.cl| (\textit{kernel}) yang akan melakukan Gaussian \textit{blurring} dengan CPU dan GPU serta menghitung waktu eksekusinya.

\begin{figure}
	\centering
	\includegraphics[width=1.0\textwidth]
	{pics/gaussian_blur_cpu_gpu.png}
	\caption{Kinerja Gaussian Blurring CPU vs GPU}
	\label{fig:gaussian_blur_cpu_gpu}
\end{figure}

Hasil eksperimen yang dilakukan menggunakan gambar BMP dengan ukuran 0,2 MB - 38,16 MB memberikan hasil seperti grafik pada gambar \ref{fig:gaussian_blur_cpu_gpu}. Pada hasil eksperimen tersebut terlihat bahwa pada gambar BMP dengan ukuran di bawah 3,15 MB, CPU (Intel Core i7 5500U) dapat melakukan Gaussian \textit{blurring} lebih cepat dari GPU (NVIDIA 940M). Tetapi ketika gambar BMP yang diproses sudah lebih besar dari 3,15 MB, terlihat bahwa GPU dapat menyelesaikan proses dengan lebih cepat. Bahkan untuk ukuran gambar paling besar (38,16 MB), waktu yang dibutuhkan GPU untuk melakukan blurring hanya 0,4 detik atau sekitar $\nicefrac{1}{5}$ dari waktu yang dibutuhkan CPU.
%---------------------------------------------------------------
\chapter{\kontribusi}
%---------------------------------------------------------------
Kontribusi tiap anggota kelompok pada tugas ini adalah sebagai berikut: \\[8pt]

\bo{Muhammad Fathurachman}:
\begin{itemize}
	\item Kajian dan eksperimen pemrograman paralel R
\end{itemize}

\bo{Otniel Yosi Viktorisa}:
\begin{itemize}
	\item Kajian pemrograman paralel R
\end{itemize}

\bo{Yohanes Gultom}:
\begin{itemize}
	\item Kajian dan eksperimen pemrograman OpenCL
\end{itemize}



%
% Daftar Pustaka
%%
% Daftar Pustaka 
% 

% 
% Tambahkan pustaka yang digunakan setelah perintah berikut. 
% 
\begin{thebibliography}{4}

\bibitem{r.patric}
{Patric. \f{Accelerate R Applications with CUDA}. 2014. \url{https://devblogs.nvidia.com/parallelforall/accelerate-r-applications-cuda/}.}

\bibitem{r.venables}
{W. N. Venables dan D. M. Smith. \f{Introduction to R}. 2016.}

\bibitem{r.beckmw}
{Beckmw. \f{A brief foray into parallel processing with R}. 2014. }

\bibitem{r.leach}
{Clint Leach. \f{Introduction to Parallel Computing in R}. 2014. }

\bibitem{r.schmidt}
{Drew Schmidt. \f{High Performance Computing with R}. 27 Februari 2015. }

\bibitem{r.schwendinger}
{Florian Schwendinger, Gregor Kastner, Stefan Theußl. \f{High Performance Computing with Applications in R}. 28 September 2015. }

\bibitem{opencl.howto}
{Askubuntu.com. \f{How to make OpenCL work on 14.10 + Nvidia 331.89 drivers?}. 10 Oktober 2014. \url{http://askubuntu.com/questions/541114/how-to-make-opencl-work-on-14-10-nvidia-331-89-drivers}.}

\bibitem{opencl.mmul}
{Zaius. \f{Matrix Multiplication 1 (OpenCL)}. 22 September 2009. \url{http://gpgpu-computing4.blogspot.co.id/2009/09/matrix-multiplication-1.html}.}

\bibitem{opencl.mukherjee}
{Mukherjee, S et al. \f{Exploring the Features of OpenCL 2.0}. 2015. IWOCL }

\bibitem{opencl.ebook}
{Banger, R, Bhattacharyya .K. \f{OpenCL Programming by Example}. 2013. Packt publishing }

\bibitem{opencl.clgetplatforminfo}
{Stackoverflow. \f{What is the right way to call clGetPlatformInfo?}. 21 Juni 2013. \url{http://stackoverflow.com/questions/17240071/what-is-the-right-way-to-call-clgetplatforminfo}.}

\bibitem{blas.netlib}
{Netlib.org. \f{About BLAS}. 15 November 2015. \url{http://www.netlib.org/blas/​​}.}

\bibitem{blas.wiki}
{Wikipedia. \f{Basic Linear Algebra Subprograms}. 10 Mei 2016. \url{https://en.wikipedia.org/wiki/Basic_Linear_Algebra_Subprograms}.}

\bibitem{opencl.clblas}
{ClMath. \f{ClBLAS}. 2016. \url{https://github.com/clMathLibraries/clBLAS}.}

\bibitem{opencl.clblas.tutorial}
{Nugteren, C. \f{ClBLAS Tutorial}. 2014. \url{http://www.cedricnugteren.nl/tutorial.php?page=1}.}

\bibitem{opencl.mygemm}
{Nugteren, C. \f{MyGEMM}. 2014. \url{https://github.com/cnugteren/myGEMM}.}

\bibitem{opencl.cuda.porting}
{sharcnet.ca. \f{Porting CUDA to OpenCL}. 2014. \url{https://www.sharcnet.ca/help/index.php/Porting_CUDA_to_OpenCL}.}

\bibitem{opencl.gaussianblur}
{Karapetsas, L. \f{Playing with OpenCL: Gaussian Blurring}. 2014. \url{ http://blog.refu.co/?p=663 }.}

​
\end{thebibliography}


%
% Lampiran 
%
%\begin{appendix}
	%\include{markLampiran}
	%\setcounter{page}{2}
	%\include{lampiran}
%\end{appendix}

\end{document}