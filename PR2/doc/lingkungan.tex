%-----------------------------------------------------------------------------%
\chapter{\lingkungan}
%-----------------------------------------------------------------------------%

\section{Lingkungan \f{Cluster}}

Dalam eksperimen yang dilakukan, kelompok kami menggunakan empat buah lingkungan yaitu \cluster Rocks University of California San Diego (UCSD), \cluster Fasilkom Universitas Indonesia (UI) dan \cluster Rocks pada \f{laptop}.

%-----------------------------------------------------------------------------%
\subsection{Cluster Rocks University of California San Diego (UCSD)}
%-----------------------------------------------------------------------------%
\Cluster Rocks\footnote{www.rocksclusters.org} milik University of California San Diego (UCSD) ini dapat diakses pada alamat \f{nbcr-233.ucsd.edu} menggunakan protokol SSH dari komputer yang telah didaftarkan \f{public key} nya. Berdasarkan informasi dari aplikasi \f{monitoring} Ganglia \footnote{http://nbcr-233.ucsd.edu/ganglia}, cluster ini terdiri 10 \nodes dengan total 80 prosesor. 

Pada \cluster ini sudah terpasang pustaka komputasi paralel OpenMPI\footnote{https://www.open-mpi.org/} dan MPICH\footnote{https://www.mpich.org/} serta paket dinamika molekular AMBER. Program paralel MPI dan eksperimen AMBER dijalankan mekanisme antrian \f{batch-jobs} untuk menjamin ketersediaan sumberdaya komputasi (\f{computing nodes}) saat program dieksekusi. Pengaturan eksekusi program paralel ini ditangani oleh \f{Sun Grid Engine}\footnote{http://www.rocksclusters.org/roll-documentation/sge/5.4/} yang juga tersedia dalam paket Rocks.


%-----------------------------------------------------------------------------%
\subsection{Cluster Fasilkom Universitas Indonesia (UI)}
%-----------------------------------------------------------------------------%
\Cluster milik Fakultas Ilmu Komputer (Fasilkom) UI ini berada di jaringan lokal yang tidak bisa diakses langsung dari luar. \Cluster ini berbasis Linux dan terdiri dari empat buah \nodes dengan total 32 prosesor. 

Pada \cluster ini sudah tersedia pustaka OpenMPI untuk menjalankan program paralel. Program dijalankan langsung tanpa mekanisme \f{batch-jobs} seperti pada \cluster UCSD dengan menjalankan program dari direktori khusus (supaya dapat direplikasi ke seluruh \nodes pada \cluster).

%-----------------------------------------------------------------------------%
\subsection{Cluster Rocks pada Laptop}
%-----------------------------------------------------------------------------%
Rocks merupakan distribusi Linux CentOS yang dikustomisasi untuk membangun \cluster \f{high performance computing (HPC)} yang bersifat \f{opensource}. Rocks menyediakan berbagai macam paket (\f{rolls}) yang digunakan dalam berbagai \f{task} HPC.

Dalam eksperimen ini Rocks 6.2 Sidewinder dipakai untuk membangun \cluster dengan dua buah \f{virtual machine (VM)} VirtualBox\footnote{https://www.virtualbox.org/} pada \f{laptop} dengan sistem operasi Ubuntu (Prosesor Intel Core i7-3610QM dengan memori DDR3 8 GB serta penyimpanan HDD 1 TB). Konfigurasi \nodes pada \cluster ini adalah sbb:

\begin{enumerate}
	\item \f{Front-end Node} (1 CPU, 1GB RAM, 30GB HDD) \\
	\Node ini memiliki GUI (\f{Graphical User Interface}) berperan sebagai antarmuka pengguna dan manajer dari \f{compute node}. Program HPC idealnya tidak menggunakan sumberdaya dari \node ini karena digunakan untuk menjalankan berbagai program antarmuka dan manajemen \cluster.
	\item \f{Compute Node}: 4 CPU, 1GB RAM, 30GB HDD \\
	\Node ini tidak memiliki GUI dan berperan khusus sebagai sumberdaya komputasi program HPC.
	
\end{enumerate}

\section{Lingkungan Pengembangan}

Program paralel yang digunakan di dalam eksperimen ini dibuat menggunakan bahasa C yang di-\f{compile} menggunakan pusaka OpenMPI pada sistem operasi Linux Ubuntu dan Mint.

Kode program-program dan laporan eksperimen ini kami simpan menggunakan layanan GitHub di alamat \f{https://github.com/yohanesgultom/parallel-programming-assignment}. Hal ini kami lakukan untuk mempermudah kolaborasi dalam pembuatan program dan laporan.